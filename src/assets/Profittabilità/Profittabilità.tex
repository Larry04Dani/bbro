
\documentclass[10pt,a4paper,withhypeper,normalphoto]{altacv}
\geometry{left=1.25cm,right=1.25cm,top=1.5cm,bottom=0.5cm,columnsep=1.2cm}



\usepackage{xcolor}
\usepackage{mdframed}
\usepackage[table]{xcolor}
\newmdenv[
  backgroundcolor=blue!10, 
  linecolor=blue!80!black, 
  linewidth=2pt,            
  roundcorner=10pt          
]{sectionbox}

\usepackage{paracol}
\iftutex
  \setmainfont{Lato}
\else
  \usepackage[default]{lato}
\fi

\definecolor{Gold}{HTML}{ab7244}
\definecolor{SlateGrey}{HTML}{2E2E2E}
\definecolor{LightGrey}{HTML}{666666}

\definecolor{high}{HTML}{C18E66}
\definecolor{medium}{HTML}{DDAC8B}
\definecolor{low}{HTML}{F2D7CB}
\definecolor{high_}{HTML}{F6CFB0}
\definecolor{medium_}{HTML}{FFDAC1}
\definecolor{low_}{HTML}{FFECE3}

\definecolor{azzurro}{HTML}{007EDD}

\colorlet{heading}{Gold}
\colorlet{headingrule}{Gold}
\colorlet{accent}{Gold}
\colorlet{emphasis}{SlateGrey}
\colorlet{body}{LightGrey}


\renewcommand{\cvItemMarker}{{\small\textbullet}}
\renewcommand{\cvRatingMarker}{\faCircle}



\begin{document}
\sloppy
\name{relazione di profittabilità}
\tagline{ {{INDIRIZZO}} - {{ZONA}} }
\photoR{3.5cm}{logo-trasparente}
\personalinfo{
  \email{info@bbrothersrome.it}
  \phone{+39-334-507-1226}
  \homepage{bbrothersrome.it}
}

\makecvheader

\AtBeginEnvironment{itemize}{\small}

\columnratio{0.6}

\begin{paracol}{2}

\cvsection{Stato immobile}

\cvevent{Tipologia e spazi}{}{}{}
{{TAGLI}}

\divider

\cvevent{Ubicazione}{{{ZONA}}}{}{{{INDIRIZZO}}}

\divider

\cvevent{Facilities e servizi}{}{}{}
{{FACILITIES}}


\divider

\cvevent{Grandezza/posti letto attuale}{}{}{}

\begin{itemize}
\item base: {{POSTI_LETTO}}
\item extra: {{POSTI_LETTO_EXTRA}}
\end{itemize}

\cvsection{Suggerimenti ORIENTATIVI per interventi, modifiche e migliorie }

\cvevent{Modifiche strutturali}{}{}{}
{{MODIFICHE_STRUTTURALI}}

\divider

\cvevent{Migliorie mobiliari}{}{}{}
{{MIGLIORIE_IMMOBILIARI}}

\divider

\newpage








\cvsection{dati utilizzati per la stima}

\cvevent{Parametri di ricerca}{}{}{}
\begin{itemize}
\item \textbf{Camere disponibili: {{CAMERE_DISPONIBILI}} }
\item \textbf{Bagni: {{BAGNI_ANALISI}}}
\item \textbf{Posizionamento: {{POSIZIONAMENTO}}}
\item \textbf{Posti letto totali: {{POSTI_LETTO_TOTALI}}}
\end{itemize}

\cvevent{considerazioni}{}{}{}
{{CONSIDERAZIONI}}


\cvsection {Analisi di Profittabilità Come Alloggio Turistico Intero}
Considerando le caratteristiche, la qualità e il contesto in cui si trova l'immobile, unito alla sua posizione, si può stimare un pricing (applicabile a regime, quindi considerabile dopo almeno 3/6 mesi di attività, e da gestire sempre dinamicamente) come da seguente tabella:  
\begin{table}[h!]
\centering

\begin{tabular}{|c|p{4.5cm}|p{1cm}|p{1cm}|p{1cm}|}
\hline
    \rowcolor{Gold} 
    \textcolor{white}{\textbf{Stagione}} &
    \textbf{   } &
    \textcolor{white}{\textbf{Prezzo notte}} & 
    \textcolor{white}{\textbf{Base Pax}} & 
    \textcolor{white}{\textbf{extra Pax (+2)}}\\ 
\hline
    \rowcolor{high_} \cellcolor{high} Alta & 01/maggio-31/luglio \break 01/settembre-31/ottobre \break 21/dicembre - 06/gennaio   & {{PREZZO_ALTA}} & {{POSTI_LETTO}} & {{EXTRA_PAX}}  \\ 
\hline
    \rowcolor{medium_} \cellcolor{medium} Media & 01/aprile-30/aprile \break 01/agosto-31/agosto & {{PREZZO_MEDIA}} & {{POSTI_LETTO}} & {{EXTRA_PAX}} \\ 
\hline
    \rowcolor{low_} \cellcolor{low} Bassa & 07/gennaio-31/marzo \break 01/novembre-20/dicembre & {{PREZZO_BASSA}} & {{POSTI_LETTO}} & {{EXTRA_PAX}} \\ 
\hline
\end{tabular}
\label{tab:example}
\end{table}





\begin{table}[h!]
\centering
\renewcommand{\arraystretch}{1.2}
\begin{tabular}{|>{\columncolor{high_}}c|>{\columncolor{high_}}p{1.5cm}|>{\columncolor{medium_}}c|>{\columncolor{medium_}}p{1.5cm}|>{\columncolor{low_}}c|>{\columncolor{low_}}p{1.5cm}|}
\hline
\multicolumn{2}{|>{\columncolor{high}}c|}{\textbf{Alta}} & 
\multicolumn{2}{|>{\columncolor{medium}}c|}{\textbf{Media}} & 
\multicolumn{2}{|>{\columncolor{low}}c|}{\textbf{Bassa}} \\ 
\hline
GG Totali & Occup. stimata & GG Totali & Occup. stimata & GG Totali & Occup. stimata \\ 
\hline
170 & {{DAYS_ALTA}} & 61 & {{DAYS_MEDIA}} & 134 & {{DAYS_BASSA}} \\ 
\hline
Rate (\%) & {{PERC_ALTA}}\% & Rate (\%) & {{PERC_MEDIA}}\% & Rate (\%) & {{PERC_BASSA}}\% \\ 
\hline
\multicolumn{2}{|>{\columncolor{medium}}c|}{\textbf{Totale annuo: 365}} & 
\multicolumn{2}{|>{\columncolor{medium}}c|}{\textbf{Occup. totale : {{TOTAL_DAYS}}}} & 
\multicolumn{2}{|>{\columncolor{medium}}c|}{\textbf{Occup. media: {{AVG_OCCUPANCY}}\%}} \\ 
\hline
\end{tabular}
\end{table}




\newpage



\switchcolumn

\cvsection{Introduzione}
\begin{quote}
Per valutare il potenziale di profittabilità del suo immobile in Roma abbiamo condotto un’analisi preliminare basata sugli annunci attivi delle OTA in zona e su stime immobiliari generali.

\end{quote}

\cvsection{Setup iniziale}

\cvachievement{\faTrophy}{Tipologia “Casa Vacanze” (assimilabile anche a B\&B ed Affittacamere) }{}
\begin{itemize}
    \item \ Verifica ed analisi documentale di fattibilità Catastale, \textbf{con produzione della Relazione Asseverata} sulla Casa da parte di Tecnico autorizzato (Arch., Ing., Geom.), costo circa \textbf{€800}
    \item \textbf{Presentazione della S.C.I.A.} (Segnalazione Certificata di Inizio Attività) presso il Comune, in via telematica allo Sportello Unico per le Attività Ricettive (DPR 28/12/2000, n. 445), costo \textbf{€100}
    \item \textbf{Diritti Comunali di istruttoria} per strutture ricettive entro i mq. 250, costo \textbf{€250}
\end{itemize}
\divider


\cvachievement{\faTrophy}{Tipologia “Alloggio Turistico” \break\textcolor{Gold}{(Tipologia prevista)} }{}




\begin{itemize}
    \item \textbf{Presentazione Comunicazione di Inizio Attività (CIA)} a Comune di Roma e Regione Lazio (via PEC), costo (esclusa attivazione SPID per accesso alla Pubblica Amministrazione, se richiesto) costo \textbf{€150} incluso nella Onboarding Fee nel caso di mandato Full Service 
    \item \textbf{Diritti Comunali} – Reversale CIA: da pagare online/SPID, costo \textbf{€50} (IVA inclusa, commiss. Carta €1,50 esclusa) 

\end{itemize}
\divider


\cvachievement{\faTrophy}{Per entrambe le tipologie }{}
\begin{itemize}
    \item Polizza Assicurativa per Responsabilità Civile specifico per CAV/B\&B: premio per il primo anno al costo di circa € 200 (variabile proporzionalmente in base alla estensione della proprietà ed alla copertura richiesta/massimali) 

    \item Servizio fotografico professionale (20/25 foto) degli ambienti della casa. 
    
    {\fontsize{7}{16}\selectfont Eventuali costi variabili per interventi strutturali necessari, o migliorie mobiliari o aggiunta di servizi e forniture suggeriti rimangono a carico dei proprietari, come anche i costi vivi di tasse ed utenze}
\end{itemize}



\newpage

\cvsection{I vantaggi della gestione integrata B\&Brother}


\begin{mdframed}[backgroundcolor=Gold]
    \begin{itemize}
        \item La casa sarà sempre ed immediatamente libera e disponibile per il Proprietario, per brevi periodi o per la vendita (ad esempio). 
        \item L’Affitto Breve non presenta rischi di occupazione abusiva o morosità. 
        \item I profitti da gestione in Affitto Breve sono normalmente piu’ alti di quelli per Affitti Standard, perche’ si affitta per notte e non per periodi mensili, quindi si aumenta l’incasso totale del mese frammentandolo. 


        \item La casa e’ sempre pulita, manutenuta e funzionante al 100\%, e’ una casa attiva e non ferma, e soprattutto sempre produttiva. 


        \item La gestione B\&Brother abbraccia diversi gradi di coinvolgimento del Proprietario in base alla sua volontà: puo’ essere coinvolto in tutti i processi di gestione oppure puo’ seguire “da lontano”, se vuole, e ricevere mensilmente i ricavi lasciando a noi tutta la libertà di gestione. 


        \item Infine, la gestione B\&Brother e’ assolutamente trasparente, chiara e precisa: il Proprietario puo’ intervenire o chiedere chiarimenti quando vuole, per controllo o modifica dati e/o operativo. 

    \end{itemize}
\end{mdframed}

\cvsection{ costi di gestione B\&Brother }



\begin{itemize}
    \item Onboarding Fee annuale per servizio di ritenuta di imposta prevista dal Property Managment: 200€ 

    \item Creazione e pubblicazione online di un sito web dedicato, che aumenterà il numero di prenotazioni dirette all’ appartamento: 200€ una tantum oppure gratuita con contratto di 1 anno.

    \divider

    \cvachievement{\faUsers}{Commissione operativa mensile B\&Brother}

    \begin{center}
    \cvtag{\cvevent{\textbf{ {{COMMISSIONE}}\% } }{}{}{} } \break\break
    dell’importo NETTO pagato dagli ospiti \break\break
    \textbf{la commissione include:}
    \end{center}

    \cvtag{Impostazione del processo di gestione e di pagamenti}
    \cvtag{Gestione e comunicazione con gli ospiti}\\
    \cvtag{Servizio di accoglienza ospiti (Check-in/Check-out)}
    \cvtag{Comunicazione soggiorno ospiti alla PS}
    \cvtag{Ritiro tassa di soggiorno (ove necessario)}
    \cvtag{Assistenza 24/7 agli ospiti durante il soggiorno}\\
    \cvtag{Pulizia e lavanderia (costi a carico degli ospiti)}
    \cvtag{Piccola manutenzione (costi a carico proprietario)}
    \cvtag{Assistenza ordinaria/straordinaria}
    \break 
    {\fontsize{7}{16}\selectfont viene garantita la nostra assistenza/presenza 24/7 in caso di necessità di interventi di manutenzione, e/o coordinamento tecnici/fornitori, inclusa nella commissione. }
    
    

\end{itemize}



\end{paracol}

\newpage

\cvsection{Conclusioni}

Tenendo conto di quanto sopra, della variazione di prezzo in base alla strategia apppplicata, oltre al periodo di avviamento, e mantenendo la stima abbastanza prudente per valutare la REDDITIVITA’ MINIMA potenziale, abbiamo applicato i seguenti parametri:  

                    \divider

\begin{itemize}
\item 
Percentuale di occupazione {{AVG_OCCUPANCY}}\% (equivalente a circa {{TOTAL_DAYS}} notti/anno), ponderata sulle diverse stagionalità (come da tabella precedente e relativa nota)  
\item 
Tariffa stagionale della Base Ospiti stabilita in numero di \textbf{ {{POSTI_LETTO}} } ospiti (“Base Pax” in tabella)
\end{itemize}

                    \divider

Si può generare un lordo mensile di circa  \cvtag{€ {{LORDO_MENSILE}}} \par

    \vspace{0.5cm}  

Il fatturato lordo minimo, stimato per il periodo di un anno, è di circa   \cvtag{€ {{LORDO_TOTALE}}} 

    \vspace{0.5cm}  


L'incasso netto stimato sulla base di queste informazioni, togliendo dal lordo le commissioni del portale(booking o air bnb), la commissione per la gestione del {{COMMISSIONE}}\% e le tasse da pagare con cedolare secca (direttamente da noi per suo conto) è di: \cvtag{€ {{NETTO_ESTIMATED}}}

    \vspace{0.5cm}  

per una media mensile di: \cvtag{€ {{NETTO_MENSILE}}}

                    \divider

\cvevent{Panoramica dei costi}{}{}{}

\hspace*{-1em}
\wheelchart{1.5cm}{0.5cm}{%
  {{PERC_GESTIONE}}/13em/accent!60/Costi di gestione,
  {{PERC_PORTALE}}/9em/accent!30/Commissione del portale,
  {{PERC_TASSE}}/11em/accent!10/Tasse,
  {{PERC_NETTO}}/11em/accent!90/Netto
}

                    \divider

Questa relazione è valida per l’attività di Affitto Breve in tipologia Alloggio Turistico della vostra proprietà, gestita in modalità Property Management da parte di staff qualificato di B\&Brother, e prospetta al proprietario una valutazione di uscite ed entrate al netto dei costi di gestione dell’attività vera e propria: i Ricavi netti calcolati non comprendono IMU, TARI/TARSI, bollette dei consumi, quote condominiali, costi di manutenzione ordinaria e straordinaria (esclusi i costi di avviamento summenzionati),  costi di ri-approvvigionamento (olio, sale, carta igienica, lampadine fulminate, etc.) e Welcome Kit (in caso di scelta).  

Questa Relazione, la cui validità è di 15 giorni, è stata redatta in data odierna e rispecchia la situazione di mercato attuale. 

In attesa di vostro cortese riscontro, porgiamo cordiali saluti.  

Roma, {{DATA_OGGI}}  

\end{document}
